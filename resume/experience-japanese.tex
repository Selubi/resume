%-------------------------------------------------------------------------------
%	SECTION TITLE
%-------------------------------------------------------------------------------
\cvsection{職歴}


%-------------------------------------------------------------------------------
%	CONTENT
%-------------------------------------------------------------------------------
\begin{cventries}
  \cventry
    {システムインフラエンジニア (正社員)} % Job title
    {楽天グループ株式会社} % Organization
    {東京、日本} % Location
    {2025/1 - 現在} % Date(s)
    {
      追加職務内容:
      \begin{cvbullets}
        \item 様々なステークホルダー間で調整を行いながら、日本語と英語を話すチームでのプロジェクトを主導しました。
      \end{cvbullets}
      主な成果:
      \begin{cvbullets}
        \item サーバー廃止プロセスの標準化・最適化を目指し、クロスチームでの取り組みを主導しました。データセンターおよびネットワークチームと密接に連携することで、2000台以上のレガシーサーバーの廃止に成功し、運用効率の大幅な改善と月々数百万円のコスト削減に貢献しました。
        \item チーム内でのメンターシップと知識共有を促進し、スキル向上を推進するとともに、協力と継続的改善の文化を育みました。
      \end{cvbullets}
    }
%---------------------------------------------------------

  \cventry
    {アソシエイトシステムインフラエンジニア (正社員)} % Job title
    {} % Organization
    {} % Location
    {2023/4 - 2024/12} % Date(s)
    {
        職務内容:
        \begin{cvbullets}
        \item Python、Bash、Jenkinsを使用して、50,000台以上のLinuxベースのベアメタルサーバーのプロビジョニングおよび管理を効率化する自動化パイプラインの設計および実装を行いました。
        \item Chef Infraおよび独自のツールを使用して、プライベートクラウドのベアメタル・アズ・ア・サービス(BMaaS)プラットフォームのプロビジョニングインフラを維持・最適化し、高い可用性と信頼性を確保しました。
        \item 日本および海外のチームと、異なるタイムゾーンを超えて効果的に協力しました。
        \end{cvbullets}
        主な成果:
        \begin{cvbullets}
        \item サーバープロビジョニングエラーを30%から10%に削減する、ハードウェアおよびネットワークの健康チェックやファームウェアのアップグレードを自動化するパイプラインを含む複数の新しい自動化パイプラインを設計・実装しました。
        \item 重要なターゲットを優先し、非本質的なターゲットを後回しにすることで、日々の設定同期時間を3~6時間から10~40分に短縮することを一例として、既存の複数のパイプラインを最適化しました。
        \item サーバーの廃止プロセスを独自に再定義し、効率化することで、300台以上のレガシーサーバーを退役させ、新しいハードウェアのために20台以上のラックスペースを確保しました。
        \end{cvbullets}
    }

%---------------------------------------------------------
  \cventry
    {バックエンドエンジニア (インターン)} % Job title
    {株式会社マネーフォワード} % Organization
    {リモート、日本} % Location
    {2022/4 - 2022/5} % Date(s)
    {
        Goマイクロサービスにおけるカスケード障害の防止を強化するために、サーキットブレーカーを開発。
    }

%---------------------------------------------------------
  \cventry
    {スタディーアシスタント} % Job title
    {関西学院大学} % Organization
    {日本} % Location
    {2021年4月 - 2021年9月} % Date(s)
    {
      \href{https://www.kwansei.ac.jp/education/ai}{AI活用人材育成プログラム}の授業において、学生の開発環境構築の支援や、クライアントサーバーAPIアーキテクチャに関する個別指導を行いました。
    }

% %---------------------------------------------------------
%   \cventry
%     {数学講師 (非常勤)} % Job title
%     {大阪文化国際学校} % Organization
%     {大阪、日本} % Location
%     {2021/2 - 2021/11} % Date(s)
%     {
%       \begin{cvitems} % Description(s) of tasks/responsibilities
%         \item {30名ほどの留学生クラスの日本留学試験(EJU)の数学準備を担当しました。}
%         \item {外国と日本の数学の記号や解き方などを注目したカリキュラムをゼロから作成しました。}
%         \item {前年度より平均点数が10\%, 最高点数が20\% 上昇しました}
%       \end{cvitems}
%     }

%---------------------------------------------------------
\end{cventries}
