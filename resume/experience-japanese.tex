%-------------------------------------------------------------------------------
%	SECTION TITLE
%-------------------------------------------------------------------------------
\cvsection{職歴}


%-------------------------------------------------------------------------------
%	CONTENT
%-------------------------------------------------------------------------------
\begin{cventries}

%---------------------------------------------------------
  \cventry
    {アソシエイトインフラエンジニア (正社員)} % Job title
    {楽天グループ株式会社} % Organization
    {東京、日本} % Location
    {2023/4 - 現在} % Date(s)
    {
      \begin{cvjobdesc} % Description(s) of tasks/responsibilities
        私は、プライベートクラウド内でミッションクリティカルなベアメタルサービス(BMaaS)を提供するDevOpsチームの一員です。このサービスは、LBaaSやCaaSのような他のクラウドサービスの基盤として機能しています。日々の業務では、
        自動化パイプラインの作成や必要なサービスのオーケストレーションの運用を行っています。
        \vspace{2.0mm}
        \\職務内容:
        \begin{cvbullets}
        \item {自社ツールとChef Infraのような外部ソリューションを組み合わせて、数万台規模のLinuxベースのコンピュートリソースのプロビジョニング、運用、およびトラブルシューティングを行います。}
        \item {Python、Bash、RubyスクリプティングとJenkinsなどのCI/CDツールやカスタムIaCを使用して、自動化パイプラインを設計・実装します。}
        \item {異なる国々のメンバーからなる日本チーム内だけでなく、異なるタイムゾーンにいる海外チームとも効果的に協力します。}
        \end{cvbullets}
        \vspace{2.0mm}
        主要プロジェクト:
        \begin{cvbullets}
        \item {ファームウェアアップグレードの標準化および自動化のエンドツーエンド実装。}
        \item {サーバーリサイクルプロセスの検証パイプラインの効率化。}
        \end{cvbullets}
      \end{cvjobdesc}
    }

%---------------------------------------------------------
\cventry
  {バックエンドエンジニア (インターン)} % Job title
  {株式会社マネーフォワード} % Organization
  {リモート、日本} % Location
  {2022/4 - 2022/5} % Date(s)
  {
    \begin{cvitems} % Description(s) of tasks/responsibilities
      \item {Goマイクロサービスにおけるカスケード障害の防止を強化するために、サーキットブレーカーを開発しました。}
    \end{cvitems}
  }

%---------------------------------------------------------
\cventry
  {スタディーアシスタント} % Job title
  {関西学院大学} % Organization
  {日本} % Location
  {2021年4月 - 2021年9月} % Date(s)
  {
    \begin{cvjobdesc} % Description(s) of tasks/responsibilities
      \href{https://www.kwansei.ac.jp/education/ai}{AI活用人材育成プログラム}の授業で教授を手伝いました。
      \vspace{2.0mm}
      \begin{cvbullets}
        \item {学生の開発環境の構築を支援しました。}
        \item {クライアントサーバーAPIアーキテクチャに関する個別指導を行いました。}
      \end{cvbullets}
    \end{cvjobdesc}
  }

%---------------------------------------------------------
  \cventry
    {数学講師 (非常勤)} % Job title
    {大阪文化国際学校} % Organization
    {大阪、日本} % Location
    {2021/2 - 2021/11} % Date(s)
    {
      \begin{cvitems} % Description(s) of tasks/responsibilities
        \item {30名ほどの留学生クラスの日本留学試験(EJU)の数学準備を担当しました。}
        \item {外国と日本の数学の記号や解き方などを注目したカリキュラムをゼロから作成しました。}
        \item {前年度より平均点数が10\%, 最高点数が20\% 上昇しました}
      \end{cvitems}
    }

%---------------------------------------------------------
\end{cventries}
