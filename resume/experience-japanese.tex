%-------------------------------------------------------------------------------
%	SECTION TITLE
%-------------------------------------------------------------------------------
\cvsection{職歴}


%-------------------------------------------------------------------------------
%	CONTENT
%-------------------------------------------------------------------------------
\begin{cventries}
  \cventry
    {システムインフラエンジニア (正社員)} % Job title
    {楽天グループ株式会社} % Organization
    {東京、日本} % Location
    {2025/1 - 現在} % Date(s)
    {
      主な職務内容:
      \begin{cvbullets}
        \item 日本語と英語を使用する国内外のチームを率い、クロスファンクショナルなプロジェクトを主導。多様なステークホルダーとの調整を行い、円滑なプロジェクト推進を実現。
        \item データセンター(DC)およびネットワーク(NW)チームと緊密に連携し、予測されるコンピューティング需要に対応するための老朽サーバーリプレースプロジェクトを推進。新規サーバー導入を円滑に進めるため、以下の関連業務を主導的に担当:
          \begin{itemize} 
            \item 【要件定義・評価】廃棄対象となる老朽サーバーの特定、および関連システムへの影響分析・評価を実施し、廃棄の可否を判断。
            \item 【設計】老朽サーバーの安全な論理廃棄プロセス(データ処理、システムからの切り離し等)および物理廃棄依頼フロー(DC/NWチーム連携含む)を設計。
            \item 【設定・実行】設計に基づいた老朽サーバーの論理廃棄作業を自身で実行。DC/NWチームへの物理廃棄依頼および、その作業進捗を管理。
            \item 【テスト・検証】サーバー廃棄後、関連システムが正常に稼働していること、および意図しない影響がないことを検証。
            \item 【標準化・ナレッジ共有】本プロセスを標準化し、誰でも作業を再現できるよう詳細な手順書を作成。ジュニアメンバーを含むチーム内への積極的なメンタリングやトレーニングを通じて、属人化の排除とチーム全体のスキル向上に貢献。
          \end{itemize}
        \item 将来のハードウェア(HW)およびソフトウェア(SW)検証プロセスを効率化・高度化するための専用ラボ環境構築プロジェクトを企画段階から推進。以下の業務担当:
          \begin{itemize}
            \item 【計画・リソース調達】既存データセンターで管理されている資産の中から、検証ラボ用ハードウェア(サーバー、ネットワーク機器等)の評価・選定を実施。新設ラボへの移行計画を策定し、データセンターチームの機器廃棄プロセスと連携しながら実行。
            \item 【物理インフラ構築】ラボ環境の物理セットアップ全般を担当。サーバーやスイッチのラックマウント・アンマウント、最適なケーブリング(配線設計および接続作業)、電源系統の整備を行い、実運用可能な物理基盤を構築。
            \item 【基盤システム構築】検証作業の基盤となる簡易Infrastructure Management System (IMS) の設計・デプロイを実施。DHCPサーバー、DNSサーバー、インターネットゲートウェイ(ルーター)、およびセキュアなリモートアクセス環境をセットアップし、ラボ利用者のアクセシビリティと利便性を確保。
            \item 【運用準備・ドキュメント化】ラボの運用手順書(初期版)および構成情報をドキュメント化し、チーム内でのナレッジ共有と将来の円滑な運用体制への移行準備を実施。
          \end{itemize}
        \item 内部インフラコンポーネント(例:内部プロビジョニングサーバー)との連携を実現する、20万行以上のPythonコードおよび設定ファイルで構成された内部自動化フレームワークリポジトリを所有し、維持・管理しています。このフレームワークは複数のチームによって利用され、業務の効率化に貢献しています。
      \end{cvbullets}
      主な成果:
      \begin{cvbullets}
        \item サーバー廃止プロセスの標準化・最適化を目指し、クロスチームでの取り組みを主導しました。データセンターおよびネットワークチームと密接に連携することで、2000台以上のレガシーサーバーの廃止に成功し、運用効率の大幅な改善と月々数百万円のコスト削減に貢献しました。
        \item 厳しい納期に対応し、老朽サーバーの特定と撤去を迅速に行い、新規サーバー導入のためのラックスペースを確保しました。電力供給が限られる中でも、将来的なコンピューティング需要を満たす運用を実現しました。
        \item チーム内でのメンターシップと知識共有を促進し、スキル向上を推進するとともに、協力と継続的改善の文化を育みました。
      \end{cvbullets}
    }
%---------------------------------------------------------

  \cventry
    {アソシエイトシステムインフラエンジニア (正社員)} % Job title
    {} % Organization
    {} % Location
    {2023/4 - 2024/12} % Date(s)
    {
        職務内容:
        \begin{cvbullets}
        \item Python、Bash、Jenkinsを使用して、50,000台以上のLinuxベースのベアメタルサーバーのプロビジョニングおよび管理を効率化する自動化パイプラインの設計および実装を行いました。
        \item Chef Infraおよび独自のツールを使用して、オンプレミスプライベートクラウドのベアメタル・アズ・ア・サービス(BMaaS)プラットフォームのプロビジョニングインフラを維持・最適化し、高い可用性と信頼性を確保しました。
        \item 日本および海外のチームと、異なるタイムゾーンを超えて効果的に協力しました。
        \end{cvbullets}
        主な成果:
        \begin{cvbullets}
        \item サーバープロビジョニングエラーを30%から10%に削減する、ハードウェアおよびネットワークの健康チェックやファームウェアのアップグレードを自動化するパイプラインを含む複数の新しい自動化パイプラインを設計・実装しました。
        \item 重要なターゲットを優先し、非本質的なターゲットを後回しにすることで、日々の設定同期時間を3~6時間から10~40分に短縮することを一例として、既存の複数のパイプラインを最適化しました。
        \item サーバーの廃止プロセスを独自に再定義し、効率化することで、300台以上のレガシーサーバーを退役させ、新しいハードウェアのために20台以上のラックスペースを確保しました。
        \end{cvbullets}
    }

%---------------------------------------------------------
  \cventry
    {バックエンドエンジニア (インターン)} % Job title
    {株式会社マネーフォワード} % Organization
    {リモート、日本} % Location
    {2022/4 - 2022/5} % Date(s)
    {
        Goマイクロサービスにおけるカスケード障害の防止を強化するために、サーキットブレーカーを開発。
    }

%---------------------------------------------------------
  \cventry
    {スタディーアシスタント} % Job title
    {関西学院大学} % Organization
    {日本} % Location
    {2021年4月 - 2021年9月} % Date(s)
    {
      \href{https://www.kwansei.ac.jp/education/ai}{AI活用人材育成プログラム}の授業において、学生の開発環境構築の支援や、クライアントサーバーAPIアーキテクチャに関する個別指導を行いました。
    }

% %---------------------------------------------------------
%   \cventry
%     {数学講師 (非常勤)} % Job title
%     {大阪文化国際学校} % Organization
%     {大阪、日本} % Location
%     {2021/2 - 2021/11} % Date(s)
%     {
%       \begin{cvitems} % Description(s) of tasks/responsibilities
%         \item {30名ほどの留学生クラスの日本留学試験(EJU)の数学準備を担当しました。}
%         \item {外国と日本の数学の記号や解き方などを注目したカリキュラムをゼロから作成しました。}
%         \item {前年度より平均点数が10\%, 最高点数が20\% 上昇しました}
%       \end{cvitems}
%     }

%---------------------------------------------------------
\end{cventries}
