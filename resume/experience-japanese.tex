%-------------------------------------------------------------------------------
%	SECTION TITLE
%-------------------------------------------------------------------------------
\cvsection{職歴}


%-------------------------------------------------------------------------------
%	CONTENT
%-------------------------------------------------------------------------------
\begin{cventries}

%---------------------------------------------------------
  \cventry
    {アソシエイトインフラエンジニア (正社員)} % Job title
    {楽天グループ株式会社} % Organization
    {東京、日本} % Location
    {2023/4 - 現在} % Date(s)
    {
      \begin{cvjobdesc} % Description(s) of tasks/responsibilities
        職務内容:
        \begin{cvbullets}
        \item Python、Bash、Jenkins を活用した自動化パイプラインの設計・実装により、4万台以上のLinuxベースのベアメタルサーバーのプロビジョニングと管理を効率化しました。
        \item Chef Infraと社内ツールを用いて、Bare-Metal as a Serviceのプロビジョニングインフラのメンテナンスとトラブルシューティングを実施しました。
        \item 複数の職能、多言語にわたるプロジェクトを主導し、日本語話者と英語話者のチームメンバー間のコミュニケーションの架け橋として活躍しました。
        \item 多様な日本のチーム、および異なるタイムゾーンにいる海外のチームと効果的に連携しました。
        \end{cvbullets}
        主要プロジェクト:
        \begin{cvbullets}
        \item サーバーのプロビジョニングエラーを30\%から10\%に削減する、サーバーのハードウェアとネットワークのヘルスチェックとファームウェアのアップグレードを自動化するパイプラインなど、さまざまな新しいパイプラインを設計および実装しました。
        \item 日次の構成同期時間を 3~6 時間から 10~40 分に短縮するなど、既存のさまざまなパイプラインを最適化しました。
        \item 関係者と協力して、サーバーの廃止プロセスを作成し、標準化しました。300台以上のレガシーサーバーを廃止し、新しいハードウェア用に 20 ラック以上のスペースを確保しました。
        \end{cvbullets}
      \end{cvjobdesc}
    }

%---------------------------------------------------------
  \cventry
    {バックエンドエンジニア (インターン)} % Job title
    {株式会社マネーフォワード} % Organization
    {リモート、日本} % Location
    {2022/4 - 2022/5} % Date(s)
    {
      \begin{cvjobdesc} % Description(s) of tasks/responsibilities
        Goマイクロサービスにおけるカスケード障害の防止を強化するために、サーキットブレーカーを開発。
      \end{cvjobdesc}
    }

%---------------------------------------------------------
% \cventry
%   {スタディーアシスタント} % Job title
%   {関西学院大学} % Organization
%   {日本} % Location
%   {2021年4月 - 2021年9月} % Date(s)
%   {
%     \begin{cvjobdesc} % Description(s) of tasks/responsibilities
%       \href{https://www.kwansei.ac.jp/education/ai}{AI活用人材育成プログラム}の授業で教授を手伝いました。
%       \vspace{2.0mm}
%       \begin{cvbullets}
%         \item {学生の開発環境の構築を支援しました。}
%         \item {クライアントサーバーAPIアーキテクチャに関する個別指導を行いました。}
%       \end{cvbullets}
%     \end{cvjobdesc}
%   }

% %---------------------------------------------------------
%   \cventry
%     {数学講師 (非常勤)} % Job title
%     {大阪文化国際学校} % Organization
%     {大阪、日本} % Location
%     {2021/2 - 2021/11} % Date(s)
%     {
%       \begin{cvitems} % Description(s) of tasks/responsibilities
%         \item {30名ほどの留学生クラスの日本留学試験(EJU)の数学準備を担当しました。}
%         \item {外国と日本の数学の記号や解き方などを注目したカリキュラムをゼロから作成しました。}
%         \item {前年度より平均点数が10\%, 最高点数が20\% 上昇しました}
%       \end{cvitems}
%     }

%---------------------------------------------------------
\end{cventries}
